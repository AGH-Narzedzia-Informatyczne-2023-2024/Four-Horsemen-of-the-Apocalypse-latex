\section{Maciej Jastrzębski}
%\label{sec:pawljmlo}

I added a photo of a famine (see Figure~\ref{fig:famine}).

\begin{figure}[htbp] % Co oznacza [htbp]?
    \centering
    \includegraphics[width=0.7\textwidth]{pictures/famine.jpg} % Jak sprawić, żeby obrazek był większy?
    \caption{This is one of the horsemen - Famine}
    \label{fig:famine}
\end{figure}

Table~\ref{tab:mytable} represents my specific numbers. % Do czego służy \ref{}?


\begin{table}[ht]
\centering
\begin{tabular}{|l|l|l|l|}
\hline
\multicolumn{1}{|c|}{1} & 2 & 3 & 4 \\ \hline
5                       & 6 & 1 & 7 \\ \hline
8                       & 9 & 0 & 0 \\ \hline
6                       & 5 & 0 & 0 \\ \hline
\end{tabular}
\label{tab:mytable}
\caption{Caption of my table}
\end{table}

Wzory: 
\[A^{-1} = \frac{1}{|A|} \cdot \left(A^D\right)^T\]
\[\lim_{x\to x^-} f(x) = \lim_{x\to x^+} f(x) = f(x)\]


To-do list (in order)
\begin{enumerate}
    \item math
    \item gym
    \item shopping
\end{enumerate}
\vspace{10pt}

Shopping list
\begin{itemize}
    \item eggs
    \item potatoes
    \item tomatoes
    \item milk
    \item[+] gift for friend Michał
\end{itemize}



\hspace{\parindent}\textbf{The Pygmalion effect}, or \textbf{Rosenthal effect}, is a psychological phenomenon in which \underline{high expectations} lead to \underline{improved performance} in a given area and \underline{low expectations} lead to \underline{worse}. The effect is named for the Greek myth of Pygmalion, the sculptor who fell so much in love with the perfectly beautiful statue he created that the statue came to life. The psychologists \textbf{Robert Rosenthal} and \textbf{Lenore Jacobson} present a view, that has been called into question as a result of later research findings, in their book \textit{Pygmalion in the Classroom}; borrowing something of the myth by advancing the idea that teachers' expectations of their students affect the students' performance. Rosenthal and Jacobson held that high expectations lead to better performance and low expectations lead to worse, both effects leading to self-fulfilling prophecy.

\hspace{\parindent}According to the Pygmalion effect, the targets of the expectations internalize their positive labels, and \textbf{those with positive labels succeed} accordingly; a similar process works in the opposite direction in the case of low expectations. The idea behind the Pygmalion effect is that \textbf{increasing the leader's expectation of the follower's performance will result in better follower performance}. Within sociology, the effect is often cited with regard to education and social class. 